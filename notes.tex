% Created 2017-07-13 Thu 15:12
\documentclass[11pt]{article}
\usepackage[utf8]{inputenc}
\usepackage[T1]{fontenc}
\usepackage{fixltx2e}
\usepackage{graphicx}
\usepackage{longtable}
\usepackage{float}
\usepackage{wrapfig}
\usepackage{rotating}
\usepackage[normalem]{ulem}
\usepackage{amsmath}
\usepackage{textcomp}
\usepackage{marvosym}
\usepackage{wasysym}
\usepackage{amssymb}
\usepackage{hyperref}
\tolerance=1000
\usepackage{minted}
\author{Rafi Khan}
\date{\today}
\title{Notes for learning C}
\hypersetup{
  pdfkeywords={},
  pdfsubject={},
  pdfcreator={Emacs 25.2.1 (Org mode 8.2.10)}}
\begin{document}

\maketitle
\tableofcontents

\section{K \& R}
\label{sec-1}
\subsection{Chapter 1 - A tutorial introduction}
\label{sec-1-1}
\begin{itemize}
\item Overview of many language feautures
\end{itemize}

\subsubsection{Hello world}
\label{sec-1-1-1}
\begin{minted}[]{c}
#include <stdio.h> // Includes io functions from the standard library

int main () {
  // Prints to standard output
  // '\n' is an escape sequence that prints a new line
  printf("hello, world\n");  // the string to be printed out is given as an argument
}
\end{minted}

An identical program can be created using \texttt{puts}
\begin{minted}[]{c}
#include <stdio.h>

int main() {

  puts("hello, world");

  return 0;
}
\end{minted}

The \texttt{printf} functions is needed for \emph{formatted} output.

\begin{enumerate}
\item Exercise 1-1
\label{sec-1-1-1-1}
Run the hello world program. Already done above

\item Exercise 1-2
\label{sec-1-1-1-2}
Use \texttt{\textbackslash{}c} where c is a non existing escape sequence

\begin{minted}[]{c}
#include <stdio.h> // Includes io functions from the standard library

int main () {
  // Prints to standard output
  // '\n' is an escape sequence that prints a new line
  printf("hello, world\y");  // the string to be printed out is given as an argument
}
\end{minted}

\begin{itemize}
\item The escape character \texttt{\textbackslash{}} is ignored and instead \texttt{y} is printed
\end{itemize}

\item {\bfseries\sffamily TODO} Research printf
\label{sec-1-1-1-3}
\end{enumerate}


\subsubsection{Fahrenheit and Celcius}
\label{sec-1-1-2}
\begin{itemize}
\item Variables
\item Arithmetic
\end{itemize}

\begin{minted}[]{c}
#include <stdio.h>

int main() {

  int lower = 0;
  int upper = 300;
  int step = 20;

  int fahr = lower;

  while(fahr <= upper) {
    int celcius = (5.0 / 9.0) * (fahr - 32);
    printf("%d\t%d\n", fahr, celcius);

    fahr += step;
  }

  return 0;
}
\end{minted}

\begin{itemize}
\item \texttt{5.0} and \texttt{9.0} are floats which evaluate to a float instead of 0 if \texttt{5} and \texttt{9} were used instead
\item All statements in C end with a semicolon
\item A while loop evaluates its condition every time and executes its instructions if true, other wise it exits
\item \texttt{5/9} in C truncates to 0
\begin{minted}[]{c}
#include <stdio.h>

int main() {

  printf("%d", 5/9); // Truncated to 0 as an integer

  return 0;
}
\end{minted}
\end{itemize}

\begin{enumerate}
\item Data types
\label{sec-1-1-2-1}
C has many data types

\begin{enumerate}
\item Research
\label{sec-1-1-2-1-1}
\begin{enumerate}
\item {\bfseries\sffamily TODO} char
\label{sec-1-1-2-1-1-1}
\item {\bfseries\sffamily TODO} short
\label{sec-1-1-2-1-1-2}
\item {\bfseries\sffamily TODO} long
\label{sec-1-1-2-1-1-3}
\item {\bfseries\sffamily TODO} double
\label{sec-1-1-2-1-1-4}
\end{enumerate}

\item Other data types
\label{sec-1-1-2-1-2}
There are also other more complex data types such as arrays, structures and unions.
\end{enumerate}
\end{enumerate}
% Emacs 25.2.1 (Org mode 8.2.10)
\end{document}
